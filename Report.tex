% !TEX TS-program = pdflatex
% !TEX encoding = UTF-8 Unicode

% This is a simple template for a LaTeX document using the "article" class.
% See "book", "report", "letter" for other types of document.

\documentclass[11pt]{article} % use larger type; default would be 10pt

\usepackage[utf8]{inputenc} % set input encoding (not needed with XeLaTeX)

%%% Examples of Article customizations
% These packages are optional, depending whether you want the features they provide.
% See the LaTeX Companion or other references for full information.

%%% PAGE DIMENSIONS
\usepackage{geometry} % to change the page dimensions
\geometry{a4paper} % or letterpaper (US) or a5paper or....
% \geometry{margin=2in} % for example, change the margins to 2 inches all round
% \geometry{landscape} % set up the page for landscape
%   read geometry.pdf for detailed page layout information

\usepackage{graphicx} % support the \includegraphics command and options

% \usepackage[parfill]{parskip} % Activate to begin paragraphs with an empty line rather than an indent

%%% PACKAGES
\usepackage{booktabs} % for much better looking tables
\usepackage{array} % for better arrays (eg matrices) in maths
\usepackage{paralist} % very flexible & customisable lists (eg. enumerate/itemize, etc.)
\usepackage{verbatim} % adds environment for commenting out blocks of text & for better verbatim
\usepackage{subfig} % make it possible to include more than one captioned figure/table in a single float
% These packages are all incorporated in the memoir class to one degree or another...

%%% HEADERS & FOOTERS
\usepackage{fancyhdr} % This should be set AFTER setting up the page geometry
\pagestyle{fancy} % options: empty , plain , fancy
\renewcommand{\headrulewidth}{0pt} % customise the layout...
\lhead{}\chead{}\rhead{}
\lfoot{}\cfoot{\thepage}\rfoot{}

%%% SECTION TITLE APPEARANCE
\usepackage{sectsty}
\allsectionsfont{\sffamily\mdseries\upshape} % (See the fntguide.pdf for font help)
% (This matches ConTeXt defaults)

%%% ToC (table of contents) APPEARANCE
\usepackage[nottoc,notlof,notlot]{tocbibind} % Put the bibliography in the ToC
\usepackage[titles,subfigure]{tocloft} % Alter the style of the Table of Contents
\renewcommand{\cftsecfont}{\rmfamily\mdseries\upshape}
\renewcommand{\cftsecpagefont}{\rmfamily\mdseries\upshape} % No bold!

%%% END Article customizations

%%% The "real" document content comes below...

\title{The Effect of Technology on Interpersonal Communication}
\author{James Igaba 15/U/5191/EVE 215005259}
%\date{} % Activate to display a given date or no date (if empty),
         % otherwise the current date is printed 

\begin{document}
\maketitle

\section{Abstract }

Recent technological advancements have had a drastic impact on the way individuals communicate. In this research, previous studies were analyzed, feld observations were conducted, and an online survey was administered to determine the level of engagement individuals have with their cell phones, other technologies and with each other in face-to-face situations. Findings suggest that technology has a negative effect on both the quality and quantity of face-to-face communication. Despite individuals’ awareness of the decrease of face-to-face communication as a result of technology, more than 62 percent of individuals observed on Makerere University main campus continue to use mobile devices in the presence of others. 

\section{1. Introduction.}

Little by little, technology has become an integral part of the way that people communicate with one another and has increasingly taken the place of face-to-face communication. Due to the rapid expansion of technology, many individuals fear that people may be too immersed in this digital world and not present enough in the real world.
Many people have expressed shared concerns regarding the overuse of technology and its impact on face-to-face communication, so much so that some restaurants in the United States of America have banned the use of mobile devices to ensure customers enjoy both their meal and their company.
Throughout this study, I sought to answer questions regarding technology usage and investigated whether technology affects face-to-face communication negatively. 

\section{2. Literature Review}

Before analyzing the effect of technology on face-to-face communication, it is important to understand the rapid growth of various technologies and their current usage throughout the country. Over the past few decades, technology usage has grown signifcantly. As of 2014, 19.5 million Ugandans own a mobile phone.

Many studies have been conducted regarding technology’s effect on social interaction and face-to-face communication since the rise of cellphone and social media usage in the late 2000s.
Recent advancements in communication technology have enabled billions of people to connect more easily with people great distances away, yet little has been known about how the frequent presence of these devices in social settings infuences face-to-face interactions.

One study examined the relationship between the presence of mobile devices and the quality of real-life, in-person social interactions. In a naturalistic field experiment, researchers found that conversations in the absence of mobile communication technologies were rated as signifcantly superior compared with those in the presence of a mobile device. People who had conversations in the absence of mobile devices reported higher levels of empathetic concern, while those conversing in the presence of a mobile device reported lower levels of empathy. 

Though much research has shown the negative effects of technology on face-to-face interaction, one study found that cell phone use in public might make individuals more likely to communicate with strangers. 
In 2011, Campbell and Kwak (2011) examined whether and how mobile communication infuences the extent to which one engages face to face with new people in public settings. By accounting for different types of cell phone uses, the study found evidence that mobile phone use in public actually facilitated talking with copresent strangers, for those who frequently rely on cell phones to get and exchange information about news.
Brignall and van Valey (2005) analyzed the effects of technology among “current cyber-youth” – those who have grown up with the Internet as an important part of their everyday life and interaction rituals. The two authors discovered that due to the pervasive use of the Internet in education, communication and entertainment, there has been a signifcant decrease in face-to-face interaction among youth. They suggest that the decrease in the amount of time youth spend interacting face-to-face may eventually have “signifcant consequences for their development of social skills and their presentation of self” (p. 337). 

\section{5. Methods}

A field observations and a survey was conducted to measure the level of engagement Makerere University students have with their cell phones, other technologies, and each other in face-to-face situations. 
The survey was administered to Makerere University students who were recruited randomly (probabilistically).Students were asked 11 questions regarding their technology use, habits, perceptions of face-to-face communication in the presence of technology, and engagement both face to face and screen to screen, which would help better answer the question of whether technology has a negative effect on face-to-face communication.
Based on the survey fndings, field observations were conducted at four highly populated areas on campus, including dining halls. Observations were conducted during heavy foot-traffc times, including in between classes and during lunch hours, when students would most likely be present and interacting with others. A variety of different interactions between other students and technology were recorded, including those texting or talking on the phone, those interacting with others, and those who did not have contact with 
devices. 

\section{6. Findings}
95 percent respondents owned a smartphone or tablet. When asked how frequently students use their cell phones, 60 percent of respondents said they use their phone more than 4 hours a day, with 18 percent of respondents admitting to more than 8 hours of usage a day. Some students (18 percent) reported that when spending time with friends or family, they always use their cell phone or tablet. The majority of students use their cell phone sometimes when they are with family or friends (74 percent), and only 8 percent of students rarely use their phone in the presence of friends and family. No respondents indicated that they never use their cell phone or tablet when spending time with friends or family. Additionally, 46 percent of respondentssaid they communicate with friends or family more frequently via technology than in person, while 26 percent said the opposite.  
Field observations yielded similar results regarding technology use and habits among Makerere University students.Of more than 200 students observed, 69 percent were using technology in one way or anotther. I found that 78 of 134 students observed alone (58 percent) were either texting or holding their phones, 21 (16 percent) were talking on the phone or wearing ear buds, and only 35 students (26 percent) were not using any technology. 
I also author found it important to observe students’ technology use and habits while with others as well. The author found that 38 of 100 students (38 percent) while with others used no technology; 62 percent were either texting, talking on the phone, or using a computer or tablet. 
In an effort to determine what impacts technology has on face-to-face communication, the survey asked students to rank the statement on a scale from strongly agree to strongly disagree: “It bothers me when my friends or family use technology while spending time with me.” Seventy-four percent of respondents said that they either agreed or strongly agreed with this statement, while only 6 percent disagreed. Among respondents, 20 percent  neither agreed nor disagreed. 
Another survey question asked students whether they believed the presence of technology, while spending time with others, affects face-to-face interpersonal communication negatively. An overwhelming 92 percent of respondents believed technology negatively affects face-to-face communication, and only 1 percent did not. Only 7 percent of respondents neither agreed nor disagreed. 
A third question regarding the impacts of technology on face-to-face communication asked students whether they noticed quality degradation in conversation amongst the presence of technology. 89 percent of respondents believed there was a degradation, only 5 percent disagreed, and 6 percent neither agreed nor disagreed. 
While conducting feld observations, similar results found evident degradation in the quality of conversation among those students using technology in the presence of others. One student, observed outside a campus building, was conversing on whatsapp video with individual on her iPhone. When a friend proceeded to join her in person,the female ignored her friend and continued her conversation on Whatsapp video call. Many students at the guild canteen ate lunch with their friends, but neglected to engage in any conversation. Instead, a large majority of the students in the dining hall sitting with others (73 percent) spent their time texting or using their computers or 
tablets. 
When asked for additional feedback regarding technology use and face-to-face communication, students provided a number of insightful responses. One student said, “I don’t like using my phone when I’m with friends in person, and I don’t like it when they use theirs, but if it is used in a way to stimulate conversation –like showing a funny video, or documenting our time together via Snapchat or photos – then I think it is acceptable.” Another student agreed, mentioning that whether technology affects face-to-face communication positively or negatively depends on how it is used. A third student shared similar opinions stating, “I don’t mind if it’s used to enhance a conversation (looking up important information or things relevant to a conversation); otherwise, it typically takes away from the experience in general as you can tell the other person(s) attention is divided and unfocused on the present moment.” 
Many respondents voiced their concerns that technology is diminishing society’s ability to communicate face to face. One student stated, “People have lost the ability to communicate with each other in face-to-face interactions,” while another respondent said, “Technology is making face-to-face communication much more diffcult because people use technology as a crutch to hide behind.” 
A third student responded, “I think technology impedes our ability to interact with people face to face,” and a fourth agreed that technology “both enhances what we share online and decreases what we say face to face.”
Other students shared sentiments that using technology to communicate is acceptable, but when used in the presence of others is disrespectful. One student responded, “I think putting away phones and technology is a sign of respect when having a conversation with someone and shows that you have their full attention. Even though it’s sometimes hard to have those times when people are not attached to their phones, I think it is more important than ever.” Many students mentioned that while spending time with friends or family, they have to make a conscious effort not to use technology.It appears that despite being aware of their own behaviors and habits regarding technology, the majority of students agree that face-to-face communication and the quality of conversations are negatively impacted by technology.

\section{7. Conclusion.}
Field observations, a survey of 50 Makerere University students, and an analysis of previously conducted studies provided evidence that the rapid expansion of technology is negatively affecting face-to-face communication. 
People are becoming more reliant on communicating with friends and family through technology and are neglecting to engage personally, uninhibited by phones and devices, even when actually in the presence of others. A majority of individuals felt the quality of their conversations degraded in the presence of technology, and many individuals were bothered when friends or family used technology while spending time together. 
Additionally, nearly half of survey respondents (46 percent) communicate more frequently with friends and family via technology than in person, indicating strongly that face-to-face interactions have decreased both in quality and in quantity.
Only time will tell what the long-term impacts of this radical shift in communication methods will yield. Will employees be less able to communicate with their employers and, therefore, less able to succeed in the workforce? Or will the new skills developed through hours of cell phone use and texting result in a workforce that is more nimble and more qualifed to multi-talk? Will Millennials be unable to communicate face to face with their children, or will the new tools available to them bring their families closer together? Will the new 
technologies bring us closer together as a community, or result in fewer actual friends and a life that is more isolated and less fulflling? With technology advancing at the speed of light and human interaction changing just as quickly, it may be impossible to predict the results. However, everyone should be aware that human interaction as was once known may have already changed forever.

\section{8. Limitations }

It is important to consider limitations to this study. The survey used a convenience sample, and therefore, cannot be generalized to a greater population. Additionally, the survey used a volunteer sample of self-selected subjects to participate in the study, potentially bringing about biases.

\section{9. References }
1) Uganda Communications Commission,
http://www.ucc.co.ug/data/qmenu/3/Facts-and-Figures.html
2) Cyber-youth by Brignall and van Valey.

\end{document}


